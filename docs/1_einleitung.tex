\section{Aufgabe der Komponente}\label{Aufgabe der Komponente}
Im Rahmen dieses Projektes soll ein neues Dateiformat entstehen, welches Styled-Layer-Descriptor(SLD)-Dateien verkürzt abbilden kann. Das SLD-Dateiformat wird von dem im OHDM-Projekt eingesetzten Geoserver benutzt, um Stile für verschiedene Objekte bzw. Ebenen (sogenannte Layer) zu definieren.  Die Definition dieser Stile erfolgt innerhalb der SLD-Dateien über XML. Da SLD eine Vielzahl an Möglichkeiten bietet, diese aber häufig nicht vollständig benötigt werden, ist der Anteil an überflüssigem Code hoch. Das in diesem Projekt entstehende Dateiformat \enquote{SLD-Light} hat die Aufgabe, den Anteil an überflüssigen Informationen zu minimieren, aber dennoch die am häufigsten benutzten Funktionalitäten zur Verfügung zu stellen.

Weiterhin wird ein Editor entwickelt, welcher dem Benutzer Möglichkeiten bietet SLD-Light-Dateien zu erzeugen, ohne dabei die Eigenschaften des Dateiformats kennen zu müssen. Das bedeutet der Editor bietet eine Benutzer-Schnittstelle mit welcher eine SLD-Light-Datei mit allen zur Verfügung stehenden Eigenschaften definiert, erzeugt und gespeichert werden kann. Außerdem bietet er Funktionalitäten um die erzeugte SLD-Light-Datei in eine SLD-Datei zu übersetzen, welche dann letztendlich vom Geoserver benutzt werden kann. Der Editor ermöglicht es, SLD-Dateien auf den Geoserver hochzuladen. Gespeicherte SLD-Light-Dateien können vom Editor wieder geöffnet werden, wodurch die Benutzeroberfläche wiederhergestellt wird, um eine einfache Veränderung der Datei zu ermöglichen.