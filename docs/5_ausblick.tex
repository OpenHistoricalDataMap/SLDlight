\section{Vorschläge / Ausblick}
Ziel dieses Projektes war es, ein neues Dateiformat zu entwickeln, welches das Erstellen von \enquote{Styled-Layer-Descriptor}-Dateien vereinfacht. Weiterhin sollte eine Anwendung erstellt werden, welche Dateien des neuen Dateiformats in SLD-Dateien überführen kann und eine grafische Benutzeroberfläche bietet, um SLD- bzw SLD-Light-Dateien zu erzeugen. 

Beide Teile wurden erfolgreich umgesetzt, wobei es trotzdem Bereiche gibt, in denen die Anwendung noch verbessert werden könnte.

Bei Anwendung des Editors in der Praxis wird sich zeigen, ob etwaige notwendige Attribute fehlen. Es wurden die Eigenschaften verwendet, welche am Häufigsten vorkommen bzw. am Wichtigsten schienen. 
Außerdem sollte die Zusammensetzung der Style-ID für die XML-Datei des Geoservers noch einmal überprüft werden. 
Weiterhin wäre es hilfreich, wenn man nicht nur eine SLD-Light-Datei mit dem Editor öffnen könnte, sondern auch eine SLD-Datei. Dies bringt allerdings andere Probleme mit sich, da die SLD-Datei Attribute definiert haben kann, die von dem Editor bzw. SLD-Light nicht unterstützt werden. 

Ergebnisformulierend ist festzustellen, dass dieses Projekt gezeigt hat, dass die Entwicklung eines neuen Dateiformats, sowie einer Anwendung, welche dieses Format übersetzen und auf dem GeoServer platzieren kann, möglich ist und dazu beiträgt die Erstellung bzw. Bearbeitung von \enquote{Styled-Layer-Descriptor}-Dateien zu vereinfachen. 